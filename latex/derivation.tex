%%% Blank Document
% \documentclass[12pt]{article}
% \usepackage{blindtext}
% \usepackage{amsmath}

% \documentclass[conference]{IEEEtran} % double column
\documentclass[journal, onecolumn]{IEEEtran} % single column
%\IEEEoverridecommandlockouts
% The preceding line is only needed to identify funding in the first footnote. If that is unneeded, please comment it out.
\usepackage{cite}
\usepackage{amsmath,amssymb,amsfonts}
\usepackage{algorithmic}
\usepackage{graphicx}
\usepackage{grffile}
\usepackage{textcomp}
\usepackage{xcolor}
\usepackage{relsize}
\usepackage[caption=false, font=footnotesize]{subfig}
\usepackage{caption}
\usepackage{float}
\usepackage[caption=false]{subfig}
\usepackage{url}
%\usepackage{IEEEtran}
\def\BibTeX{{\rm B\kern-.05em{\sc i\kern-.025em b}\kern-.08em
    T\kern-.1667em\lower.7ex\hbox{E}\kern-.125emX}}


\title{Four Wheel Steering (4WS) Control}
\author{Name: Rakshith Vishwanatha \\ ASU ID: 1217756241 \\ Mobile: 480-781-6116}
\date{\today}

\begin{document}

\maketitle

\section{Four Wheel Steer (4WS) Dynamics}
\subsection{Overview}
\subsection{Derivation for Bike}


\section{PID Controller}
\subsection{Overview}
\subsection{Derivation for Bike}


\section{Control Barrier Function}
\subsection{Overview}
\subsection{Derivation for Bike}



\section{Misc.}
\subsection{Bike Dynamics Equation}

\begin{eqnarray}
    \begin{aligned}
        (I_{bz} + I_{fz}) \ddot{\phi_b} = 
        (I_{fx} - I_{fy}) \{ 
        {\Omega}^2 {cos(\phi_w)}^2 sin(\phi_b) cos(\phi_b) \\
        -\Omega \cos(\phi_w)(\cos(2\phi_b) + I_{fz})\dot{\phi_w} \} \\
        + (m_b h_b + m_f h_f) g \sin(\phi_b)
    \end{aligned}
\end{eqnarray}

\begin{eqnarray}
    \begin{aligned}
        X &= 
        \begin{bmatrix}
            \phi_b \\
            \dot{\phi_b} \\
            \phi_w
        \end{bmatrix} \\ 
        u &= \dot{\phi_w}
    \end{aligned}
\end{eqnarray}

\begin{eqnarray}
    \begin{aligned}
        \dot{X} &= f(X) + g(X) u
    \end{aligned}
\end{eqnarray}


\begin{eqnarray}
    \begin{aligned}
        \begin{bmatrix}
            \dot{\phi_b} \\
            \ddot{\phi_b} \\
            \dot{\phi_w}
        \end{bmatrix} 
        =         
        \begin{bmatrix}
            \dot{\phi_b} \\ \\

            \frac{ (I_{fx} - I_{fy}) }{(I_{bz} + I_{fz})} 
            \{ {\Omega}^2 {cos(\phi_w)}^2 sin(\phi_b) cos(\phi_b) \} + \\
            \frac{ 1 }{(I_{bz} + I_{fz})} (m_b h_b + m_f h_f) g \sin(\phi_b) \\ \\

            0
        \end{bmatrix}
        + \\
        \begin{bmatrix}
            0 \\ \\

            \frac{ (I_{fx} - I_{fy}) }{(I_{bz} + I_{fz})} 
            \{ -\Omega \cos(\phi_w)(\cos(2\phi_b) + I_{fz})\dot{\phi_w} \} \\ \\
            
            1
        \end{bmatrix}
        \dot{\phi_w}
    \end{aligned}
\end{eqnarray}

\subsection{Control Barrier Equations}

Requirement is:
\begin{eqnarray}
    \begin{aligned}
        |\phi_b| &\leq \phi_{bMax} \\
        |\dot{\phi_b}| &\leq \dot{\phi}_{bMax} \\
        |\phi_w| &\leq \phi_{wMax}
    \end{aligned}
\end{eqnarray}

So CBFs are defined as:
\begin{eqnarray}
    \begin{aligned}
        |\phi_b| &\leq \phi_{bMax} \\
        |\dot{\phi_b}| &\leq \dot{\phi}_{bMax} \\
        |\phi_w| &\leq \phi_{wMax}
    \end{aligned}
\end{eqnarray}

This leads to siz CBF equations:
\begin{eqnarray}
    \begin{aligned}
        h_1(X):\ \ 0 &\leq \phi_{bMax} - \phi_b \\
        h_2(X):\ \ 0 &\leq \phi_{bMax} + \phi_b \\
        h_3(X):\ \ 0 &\leq \dot{\phi}_{bMax} - \dot{\phi_b}\\
        h_4(X):\ \ 0 &\leq \dot{\phi}_{bMax} + \dot{\phi_b}\\
        h_5(X):\ \ 0 &\leq \phi_{wMax} - \phi_w \\
        h_6(X):\ \ 0 &\leq \phi_{wMax} + \phi_w \\
    \end{aligned}
\end{eqnarray}

\subsection{QP Solution}

\begin{eqnarray}
    \begin{aligned}
        u^* =& \arg\min_{u \epsilon U} \frac{1}{2}||u_{des} - u_{asif} ||^2 \\
        &s.t. \ L_fh_i + L_g h_i u + \alpha h_i \geq 0
    \end{aligned}
\end{eqnarray}

The Lie derivative of a scalar function such as $h$ (as opposed to a vector function) is given by:
\begin{eqnarray}
    \begin{aligned}
        Lfh &= \triangledown h f(X) \\
            &= \Big<[\frac{\delta}{\delta x_1}, \frac{\delta}{\delta x_2}, \frac{\delta}{\delta x_3}],  h \Big> f(X) \\
            &= \Big<[\frac{\delta}{\delta \phi_b}, \frac{\delta}{\delta \dot{\phi_b}}, \frac{\delta}{\delta \phi_w}],  h \Big> f(X)
    \end{aligned}
\end{eqnarray}

Thus, the Lie derivative inequalities provided above in the squared error optimizer are:
\begin{equation}
    \begin{aligned}
        &L_fh_1 + L_g h_1 u + \alpha h_1 \geq 0 \\
        &\Big< [-1,\ 0,\ 0], h_1(X)\Big> f(X) + \Big< [-1,\ 0,\ 0], h_1(X)\Big> g(X)u + \alpha h_1 \geq 0 \\
        &-f_1(X) + 0 + \alpha h_6 \geq 0 \\
        &-\dot{\phi_b} + 0 + \alpha (\phi_{bMax} - \phi_b) \geq 0
    \end{aligned}
\end{equation}

\begin{equation}
    \begin{aligned}
        &L_fh_2 + L_g h_2 u + \alpha h_2 \geq 0 \\
        &\Big< [+1,\ 0,\ 0], h_2(X)\Big> f(X) + \Big< [+1,\ 0,\ 0], h_2(X)\Big> g(X)u + \alpha h_2 \geq 0 \\
        &+f_1(X) + 0 + \alpha h_6 \geq 0 \\
        &+\dot{\phi_b} + 0 + \alpha (\phi_{bMax} + \phi_b) \geq 0 \\
    \end{aligned}
\end{equation}

\begin{equation}
    \begin{aligned}
        &L_fh_3 + L_g h_3 u + \alpha h_3 \geq 0 \\
        &\Big< [0,\ -1,\ 0], h_3(X)\Big> f(X) + \Big< [0,\ -1,\ 0], h_3(X)\Big> g(X)u + \alpha h_3 \geq 0 \\
        &-f_2(X) - g_2(X)u + \alpha h_3 \geq 0 \\ \\ 
        & - \Bigg[
        \frac{ (I_{fx} - I_{fy}) }{(I_{bz} + I_{fz})} 
        \{ {\Omega}^2 {cos(\phi_w)}^2 sin(\phi_b) cos(\phi_b) \} + \\
        &\frac{ 1 }{(I_{bz} + I_{fz})} (m_b h_b + m_f h_f) g \sin(\phi_b) 
        \Bigg] \\
        & - \Bigg[
        \frac{ (I_{fx} - I_{fy}) }{(I_{bz} + I_{fz})} 
        \{ -\Omega \cos(\phi_w)(\cos(2\phi_b) + I_{fz})\dot{\phi_w} \} 
        \Bigg]\\
        &+ \alpha (\dot{\phi}_{bMax} - \dot{\phi}_b) \geq 0 \\
    \end{aligned}
\end{equation}

\begin{equation}
    \begin{aligned}
        &L_fh_3 + L_g h_3 u + \alpha h_3 \geq 0 \\
        &\Big< [0,\ -1,\ 0], h_3(X)\Big> f(X) + \Big< [0,\ -1,\ 0], h_3(X)\Big> g(X)u + \alpha h_3 \geq 0 \\
        &-f_2(X) - g_2(X)u + \alpha h_3 \geq 0 \\ \\ 
        & + \Bigg[
        \frac{ (I_{fx} - I_{fy}) }{(I_{bz} + I_{fz})} 
        \{ {\Omega}^2 {cos(\phi_w)}^2 sin(\phi_b) cos(\phi_b) \} + \\
        &\frac{ 1 }{(I_{bz} + I_{fz})} (m_b h_b + m_f h_f) g \sin(\phi_b) 
        \Bigg] \\
        & + \Bigg[
        \frac{ (I_{fx} - I_{fy}) }{(I_{bz} + I_{fz})} 
        \{ -\Omega \cos(\phi_w)(\cos(2\phi_b) + I_{fz})\dot{\phi_w} \} 
        \Bigg]\\
        &+ \alpha (\dot{\phi}_{bMax} + \dot{\phi}_b) \geq 0 \\
    \end{aligned}
\end{equation}

\begin{equation}
    \begin{aligned}
        &L_fh_5 + L_g h_5 u + \alpha h_5 \geq 0 \\
        &\Big< [0,\ 0,\ -1], h_5(X)\Big> f(X) + \Big< [0,\ 0,\ -1], h_5(X)\Big> g(X)u + \alpha h_5 \geq 0 \\
        &0 - g_3(X)u + \alpha h_6 \geq 0 \\
        &0 -\dot{\phi_w} + \alpha (\phi_{wMax} - \phi_w) \geq 0 \\
    \end{aligned}
\end{equation}

\begin{equation}
    \begin{aligned}
        &L_fh_6 + L_g h_6 u + \alpha h_6 \geq 0 \\
        &\Big< [0,\ 0,\ -1], h_6(X)\Big> f(X) + \Big< [0,\ 0,\ -1], h_6(X)\Big> g(X)u + \alpha h_6 \geq 0 \\
        &0 + g_3(X)u + \alpha h_6 \geq 0 \\
        &0 + \dot{\phi_w} + \alpha (\phi_{wMax} + \phi_w) \geq 0 \\
    \end{aligned}
\end{equation}


\subsection{Calculation of $\alpha$ in QP Equations}
$\alpha$ is a class Kappa function defined as:
\begin{equation}
    \begin{aligned}
        &\alpha(r) \triangleq - \min_{\begin{array} {c}\scriptstyle x \epsilon A(r) \\[-4pt]
                \scriptstyle u \epsilon U \end{array} }
        \Big( \min_{i \epsilon [i,n]} L_fh_i + L_gh_iu\Big) \\
        &\text{where,}  \\
        &A(r) \triangleq \{ x \ \epsilon \ S \ | \ 
        \exists \ i \ \epsilon \ [1, n], \ 0 \leq h_i(X) \leq r \} \subseteq S \\
        &\text{where, S is not just the safety set, but is the VIABLE safety set.}
    \end{aligned}
\end{equation}

In its expanded form this means:
$\alpha$ is a class Kappa function defined as:
\begin{equation}
    \begin{aligned}
        \label{Ar_h_ineq}
        &\alpha(r) \triangleq - \min_{\begin{array} {c}\scriptstyle x \epsilon A(r) \\[-4pt]
                \scriptstyle u \epsilon U \end{array} }
        \Big( \min 
        \begin{bmatrix}
            L_fh_1 + L_gh_1u \\
            L_fh_2 + L_gh_2u \\
            L_fh_3 + L_gh_3u \\
            L_fh_4 + L_gh_4u \\
            L_fh_5 + L_gh_5u \\
            L_fh_6 + L_gh_6u \\
        \end{bmatrix}
        \Big) \\
        \text{with the constraints placed by the $A(r)$ definition/inequality} \\
        &0 \leq h_1 \leq r \\
        &0 \leq h_2 \leq r \\
        &0 \leq h_3 \leq r \\
        &0 \leq h_4 \leq r \\
        &0 \leq h_5 \leq r \\
        &0 \leq h_6 \leq r \\
    \end{aligned}
\end{equation}

The inequalities $0 \leq h_i \leq r$ place the additional constraint/relation between 
the $\alpha$ Kappa function's $r$ and the CBF state constraints of 
(obtained by summing up similar CBF equations and simplifying):
\begin{equation}
    \begin{aligned}
        -\frac{r}{2} \leq \phi_b \leq +\frac{r}{2} \\
        -\frac{r}{2} \leq \dot{\phi}_b \leq +\frac{r}{2} \\
        -\frac{r}{2} \leq \phi_w \leq +\frac{r}{2} \\
    \end{aligned}
\end{equation}
So CBF max state values chosen should be within the overall $r/2$ bounds placed by the class Kappa function. \\

When we put the $\alpha(r)$ equation in words, it effectively means:
\begin{itemize}
    \item Create an empty 2D table of size $x \times u$ to represent all combos of states and controls
    \item The number of states itself that can be created/checked is defined by all states within the 
    viable safety set which satisfy $0 \leq h_i \leq r$ given in the $A(r)$ equation, and expanded in (\ref{Ar_h_ineq})'s inequalities.
    \item For each of the entries in the 2D matrix, calculate a $n\times 1$ vector corresponding to each of the CBF equations.
    \item Now for each of the vectors in the 2D matrix, calculate and store in-place the minimum of that vector
    \item Then, calculate the min over the whole 2D matrix
    \item Negate this value to obtain the value of $\alpha$
\end{itemize}
Ofcourse, all these steps can be vectorized and handled more efficiently using an optimizaiton toolbox like fmincon() in MATLAB.


\end{document}